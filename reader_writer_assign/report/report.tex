\documentclass{article}



\usepackage{listings}
\usepackage{color}

\lstset{numbersep=5pt,numbers=left,numberstyle=\footnotesize,title=\lstname,basicstyle=\footnotesize,showspaces=false}

\begin{document}

\title{Synchronization with monitors}

\author{Laurent Graebner,
\and Jander Nascimento, 
\and Lam }

\maketitle

\tableofcontents

\section{Implementing Thread-safe linked list}          

	\subsection{Validity of the first test}

		The first test implemented will not fail duo do its verification. The test was written
		to catch up situations where the simultaneos change would generate a conflict case and
		the programming running end up in a \textit{deadlock}.
		This test is completely valid, but the \textit{deadlock} does not happens constantly with 
		this implementation so we may say that the test is not effective.

	\subsection{Analyzing the first test output}

		Output:

		\begin{enumerate}
		      \item Thread nmbr 0, won.
		      \item Thread nmbr 1, won.
		      \item THREAD 0 TIME: +0s 0.040ms, TYPE : BEGIN WRITE
		      \item THREAD 0 TIME: +0s 0.041ms, TYPE : END READ
		      \item THREAD 0 TIME: +0s 0.041ms, TYPE : BEGIN READ
		      \item THREAD 0 TIME: +0s 0.041ms, TYPE : END READ
		      \item THREAD 0 TIME: +0s 0.041ms, TYPE : BEGIN WRITE
		      \item THREAD 0 TIME: +0s 0.041ms, TYPE : END READ
		      \item THREAD 0 TIME: +0s 0.041ms, TYPE : BEGIN READ
		      \item THREAD 0 TIME: +0s 0.042ms, TYPE : END READ
		      \item THREAD 0 TIME: +0s 0.042ms, TYPE : BEGIN WRITE
		      \item THREAD 0 TIME: +0s 0.042ms, TYPE : END READ
		      \item THREAD 0 TIME: +0s 0.042ms, TYPE : BEGIN READ
		      \item THREAD 0 TIME: +0s 0.042ms, TYPE : END READ
		      \item THREAD 0 TIME: +0s 0.042ms, TYPE : BEGIN WRITE
		      \item THREAD 0 TIME: +0s 0.042ms, TYPE : END READ
		      \item THREAD 0 TIME: +0s 0.042ms, TYPE : BEGIN READ
		      \item THREAD 0 TIME: +0s 0.043ms, TYPE : END READ
		      \item THREAD 1 TIME: +0s 0.094ms, TYPE : BEGIN WRITE
		      \item THREAD 1 TIME: +0s 0.097ms, TYPE : END READ
		      \item THREAD 1 TIME: +0s 0.097ms, TYPE : BEGIN READ
		      \item THREAD 1 TIME: +0s 0.098ms, TYPE : END READ   
		\end{enumerate}

		The consistency of the call can be analyzed looking at three parameters: the identifier of 
		the thread, type of the call and the order in which they appear in the output.

		For instance, using this output we may infer that there is two threads (and the subject says 
		that to us:), analysing the thread 0 (zero) it's possible to see that between write operations
		it's possible to have read operations.

	\subsection{Naive implementation, using mutexes}

		\lstinputlisting[language=C]{../reader_writer_1.c}

	\subsection{Naive implementation test}

		code

	\subsection{Efficiency of naive implementation}

		This kind of implementation has the same result as a serialization of all operations, since 
		only one thread can access the code at a time. That is the main reason why this kind of 
		implementation is safe.

\section{Improving concurrency}

	\subsection{Improving concurrency capability}

		\lstinputlisting[language=C]{../reader_writer_2.c}

	\subsection{Improving concurrency test}

		\lstinputlisting[language=C]{../t_improving_concurrence_test.c}

\section{Improving fairness}	

	\subsection{Fairness detection through the output}
		
\begin{lstlisting}
THREAD 0 TIME: +0s 0.354ms, TYPE : BEGIN WRITE 
THREAD 0 TIME: +0s 0.363ms, TYPE : END WRITE 
THREAD 0 TIME: +0s 0.366ms, TYPE : BEGIN READ 
THREAD 0 TIME: +0s 0.369ms, TYPE : END READ 
THREAD 0 TIME: +0s 0.373ms, TYPE : BEGIN WRITE 
THREAD 0 TIME: +0s 0.377ms, TYPE : END WRITE 
THREAD 1 TIME: +0s 0.378ms, TYPE : BEGIN WRITE 
THREAD 1 TIME: +0s 0.383ms, TYPE : END WRITE 
THREAD 1 TIME: +0s 0.386ms, TYPE : BEGIN READ 
THREAD 1 TIME: +0s 0.389ms, TYPE : END READ 
THREAD 0 TIME: +0s 0.396ms, TYPE : BEGIN READ 
THREAD 0 TIME: +0s 0.402ms, TYPE : END READ 
THREAD 0 TIME: +0s 0.406ms, TYPE : BEGIN WRITE 
THREAD 0 TIME: +0s 0.410ms, TYPE : END WRITE 
THREAD 0 TIME: +0s 0.413ms, TYPE : BEGIN READ 
THREAD 0 TIME: +0s 0.416ms, TYPE : END READ 
THREAD 0 TIME: +0s 0.420ms, TYPE : BEGIN WRITE 
THREAD 0 TIME: +0s 0.424ms, TYPE : END WRITE 
THREAD 0 TIME: +0s 0.427ms, TYPE : BEGIN READ 
THREAD 0 TIME: +0s 0.431ms, TYPE : END READ 
THREAD 0 TIME: +0s 0.434ms, TYPE : BEGIN WRITE 
THREAD 0 TIME: +0s 0.438ms, TYPE : END WRITE 
THREAD 0 TIME: +0s 0.442ms, TYPE : BEGIN READ 
THREAD 0 TIME: +0s 0.445ms, TYPE : END READ 
THREAD 0 TIME: +0s 0.449ms, TYPE : BEGIN WRITE 
THREAD 0 TIME: +0s 0.453ms, TYPE : END WRITE 
THREAD 0 TIME: +0s 0.456ms, TYPE : BEGIN READ 
THREAD 0 TIME: +0s 0.459ms, TYPE : END READ 
THREAD 0 TIME: +0s 0.463ms, TYPE : BEGIN WRITE 
THREAD 0 TIME: +0s 0.466ms, TYPE : END WRITE 
THREAD 0 TIME: +0s 0.469ms, TYPE : BEGIN READ 
THREAD 0 TIME: +0s 0.473ms, TYPE : END READ 
THREAD 0 TIME: +0s 0.476ms, TYPE : BEGIN WRITE 
THREAD 0 TIME: +0s 0.480ms, TYPE : END WRITE 
THREAD 0 TIME: +0s 0.483ms, TYPE : BEGIN READ 
THREAD 0 TIME: +0s 0.487ms, TYPE : END READ 
THREAD 0 TIME: +0s 0.490ms, TYPE : BEGIN WRITE 
THREAD 0 TIME: +0s 0.494ms, TYPE : END WRITE 
THREAD 0 TIME: +0s 0.498ms, TYPE : BEGIN READ 
THREAD 0 TIME: +0s 0.501ms, TYPE : END READ 
THREAD 0 TIME: +0s 0.505ms, TYPE : BEGIN WRITE 
THREAD 0 TIME: +0s 0.509ms, TYPE : END WRITE 
THREAD 0 TIME: +0s 0.513ms, TYPE : BEGIN READ 
THREAD 0 TIME: +0s 0.516ms, TYPE : END READ
\end{lstlisting}

As we can see in the output above, the Thread 1 just work in it's time slice in the line 7-10. All
others time slices belongs to the Thread 0, this characterize unfairness. Since two Threads do not
share the processor time in a fair manner.

\subsection{Fairness discusion}

code

\subsection{Fairness solution}

Our fair solution can be seen in the code below.



\end{document}
